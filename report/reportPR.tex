
\documentclass[10pt]{article} 

\usepackage[utf8]{inputenc} 
\usepackage{geometry} 
\usepackage{pgfplots,wrapfig}
\usepackage{amssymb}

\geometry{a4paper} 

\vspace{2cm}
\setlength{\parindent}{0cm}

\title{3D Camera Localization - Probabilistic Robotics}
\author{ Francesco Chichi}

\begin{document}
	\maketitle
	\section{Introduction}
		This project aims to reconstruct the poses of a vehicle, using a Least-Squares approach, using only the odometry and anonymized observations. 
		Using G2O simulator, we obtain a landmarks based map, the robot's perceptions and the transition of the robot between two poses (the odometry).

	\section{Implementation}
		\subsection{Data Structure}
		All the project's data structures are parsed from the G2O file by the \textit{g2o\_parser} class.
		\begin{itemize}
			\item \textbf{Landmark:} This object store the position of a landmark and his ID.
			\item \textbf{Pose:} Used to represent a robot's pose and his ID.
			\item \textbf{Transition:} This structure represent the transition of the robot from pose A to B. In it are stored the two poses and their relative ID.
			\item \textbf{Observation:} This kind of data structure is used to represent the robot's observations. 
			It contains the pose's ID, a vector of 2d landmarks and two vector with the relative ID and depth of each one.
			\item \textbf{DictPoints:} This dictionary is used to store match of 2D and 3D projection of a landmark.
		\end{itemize}
	
		\subsection{Camera}
		This class is used to represent a camera, initialized with the resolution (rows and columns), the camera matrix and the initial position in the space.
		In it we find the two principal methods\textit{projectPoints} and \textit{unprojectPoint}, used to transform a point from 2D to 3D and vice versa.
		
		\subsection{Distance Map}
		A distance map approach is used in order to compute the data associations.
		After each transition, a distance map is computed using the G2O's observations as reference and are computed the corrispondences with the projections of the landmarks observed as reference.

		\begin{figure}
			\centering
			\includegraphics[width=0.7\linewidth]{img/corrispondences}
			\caption{The green points are observed points while the red are the projected landmarks.}
			\label{fig:corrispondences}
		\end{figure}

		\subsection{Main}
		The main class is used to parse the data coming out from the G2O simulation into the main data structures of the project and, for each transition of the robot, the landmarks are filtered and only those within a range are stored and used to compute the data association, in order to use the least square to correct the predicted position of the robot.\\
		 		
	\section{Least-Squares Approach}
	\subsection{State}
	In the state there are the position of the robot in the space, represented as a rotation matrix and a transition vector:
	X $\in \textit{SE}(3)$ : X = (R|t)
	
	The increments $\Delta$x is represented as a ${\rm I\!R}^6$ vector: the position \textit{x},\textit{y},\textit{z} and the rotation among the three axes \textit{$\alpha_x$},\textit{$\alpha_y$},\textit{$\alpha_z$}.
	
	\subsection{Measurement}
	The measurements are the projections on the camera of the world's points.
	$z^{[\textit{m}]} \in {\rm I\!R}^2 : \textbf{z}^{[\textit{m}]} = (u^{[\textit{m}]} v^{[\textit{m}]})^T$
	
	\subsection{Prediction}
	The prediction of the robot's pose at time \textit{t} is obtained applying the odometry transition at time \textit{t} to the pose \textit{t-1} \textit{(current\_camera\_pose * cameraToRobot * motion * robotToCamera)}. \\
	$h^{[\textit{n}]} = proj(K X^{-1} p^{[\textit{n}]})$.
	
	\subsection{Error}
	The error is computed as the difference between the 2D projection of the landmarks, using the predicted pose, and the observation of the same:\\
	$e^{[\textit{n,m}]} = h^{[\textit{n}]} - z^{[\textit{m}]}$
	
	\subsection{Jacobian}
	For each landmarks, the column \textit{i} of the $2\times6$ jacobian is computed using the formulae of the numerical differentiation: $ \frac{f(i +  \epsilon) - f(i-\epsilon)}{2\epsilon}$.
	
	For each landmark the procedure is the same: at the \textit{i}-th element of the state \textbf{X} are summed an $\epsilon$, then is projected the point using the modified state as reference and an error \textbf{ep} is computed as the difference between this projected point and the reference one ($z^{[\textit{m}]}$);
	The same procedure is applied subtracting the $\epsilon$, computing in this way \textbf{em}.
	
	Finally the element of the state is computed as $\frac{ep - em}{2\epsilon}$
	
	\subsection{Correction}
	The correction follows the least-squares procedure, for each least-squares iteration:
	
	H $\leftarrow$ H + $J_{\textit{i}}^{T}$ $\Omega_{\textit{i}}$ $J_{\textit{i}}$	\\
	b $\leftarrow$ b + $J_{\textit{i}}^{T}$ $\Omega_{\textit{i}}$ $e_{\textit{i}}$  \\
	$\Delta x \leftarrow solve(H\Delta x=-b)$\\
	$X^* \leftarrow X^* \boxplus \Delta x$
	\section{Experimental Results}	
	\section{Front-End}
	In the \textit{Test.h} class is implemented in openCV a GUI that allows the user to visualize and navigate throw Landmarks, robot's pose and transitions.
	
	\begin{figure}
		\centering
		\includegraphics[width=0.5\linewidth]{img/exact}
		\includegraphics[width=0.8\linewidth]{img/exact2}
		\caption{The blue points are robot's poses, the black ones are the landmarks and the purple are represented the transition between two poses.}
		\label{fig:exact}
	\end{figure}
	
	
\end{document}
